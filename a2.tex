\documentclass{article}
\usepackage{graphicx} % For figures
\usepackage{float} % For figure positioning
\usepackage{subcaption} % For side-by-side figures
\usepackage{amsmath} % For math environments align, aligned, gather, gathered, multline

\begin{document}
    \raggedright
    \title{STAT380: Assignment 1  \\}
    \author{Vivienne Crowe ID:40071153}
    \date{4/10/21}
    
    \maketitle

    \section{Part A}

    \section{Part B}

    \subsection{}
    \subsection{}

    

    \subsection{}

    (a)
    $n$ is 100, $p$ is 2. Let $\epsilon \sim N[0,1]$, then the model is

    \[Y = X-2 \cdot X^2 + \epsilon\]

    (b) It looks like a quadratic relationship, values appear more dense around the origin

    (c) The LOOVC errors for the models i-iv are:
    i. 7.288
    ii. 0.937 
    iii. 0.957 
    iv. 0.954

    (d) With the new seed, the errors were the same. This was expected 
    because in the LOOCV procedure all possible subsets of size $n-1$ are 
    considered so the order that they are selected (which is random), does not 
    impacted the minimum error obtained.

    (e) The quadratic model has the smallest error, which was expected as this 
    was precisely the underlying model.

    (f) The statistical significance for the coeffecients are consistent with the 
    conclusions drawn above, p-values for the intercept, first-, and second-order terms are all 
    very small (on the order of $10^{-8}$ or less), whereas coefficient estimates for the third-, 
    and fourth-order terms are much larger, around $10^{-1}$.
 
\end{document}